\DocumentMetadata{
  lang        = en,
  pdfstandard = ua-2,
  tagging=on,
  tagging-setup={math/setup=mathml-SE} 
}

\documentclass{article}

\usepackage[T1]{fontenc}
\usepackage[english]{babel}
\usepackage[margin=0.5in, bottom=1in]{geometry}
\usepackage{color}
\definecolor{orange}{cmyk}{0,0.6,1,0.15}
\definecolor{BrickRed}{cmyk}{0,0.89,0.94,0.28}
\definecolor{DarkOrchid}{cmyk}{0.25, 0.75, 0, 0.20}
\usepackage{multicol}
\setlength{\columnsep}{3.0cm}

\usepackage{hyperref}
\hypersetup{
    pdftitle={Physics 1250 Syllabus Spring 2026},
    pdfauthor={Prof. Chris Orban},
    pdfdisplaydoctitle=true,
    pdflang={en-US}
}

\begin{document}

\begin{center}
\title{\Large Physics 1250: Mechanics, Thermal Physics, and Waves  \\ Spring Semester 2026 } \date{} \author{} \maketitle
%\textbf{Spring Semester 2026 }
\end{center}
\vspace{-1.7cm}
\noindent Course Meetings: \\
Lectures: Mondays \& Wednesdays, 8am-9:50am, Marion Science \& Engineering building room 230 \\
Lab sessions: Fridays 8am-10:05am,  Marion Science \& Engineering building room 230 \\

\noindent Final exam: 8am-9:45am, Monday, May 4 (Room: TBA) \\
Course web page: \url{http://osu.instructure.com} \\


\noindent Instructor: Prof. Chris Orban \\
Office: Marion Science \& Engineering building 210E \\
Marion Office hours: Available before or after class or by appointment \\
Contact Info: 614-557-9387, e-mail:  \href{mailto:orban@physics.osu.edu}{orban@physics.osu.edu} \\

\noindent Lab Instructor: Dr. Michael Hinton \\
Office hours: Either before or after laboratory sections. \\

\section*{GEC Course Information}

\noindent Physics 1250 and 1251 are in the Natural Science category of the General Education Curriculum (GEC). \\

\section*{Goals for GEC natural science courses}

\begin{enumerate}

\item Successful students will engage in theoretical and empirical study within the natural sciences while gaining an appreciation of the modern principles, theories, methods, and modes of inquiry used generally across the natural sciences. 

\item Successful students will discern the relationship between the theoretical and applied sciences while appreciating the implications of scientific discoveries and the potential impacts of science and technology. 

\end{enumerate}

\section*{Expected Learning Outcomes: Natural Science}

\noindent Successful students are able to

\begin{enumerate}
\item  Explain basic facts, principles, theories, and methods of modern natural sciences, and describe and analyze the process of scientific inquiry. 

\item Identify how key events in the development of science contribute to the ongoing and changing nature of scientific knowledge and methods. 

\item Employ the processes of science through exploration, discovery, and collaboration to interact directly with the natural world when feasible, using appropriate tools, models, and analysis of data. 

\item Analyze the inter-dependence and potential impacts of scientific and technological developments. 

\item Evaluate social and ethical implications of natural scientific discoveries. 

\item Critically evaluate and responsibly use information from the natural sciences. 
\end{enumerate}

\section*{How the Learning Objectives will be met in Physics 1250}

\begin{enumerate}
\item Student preconceptions and alternate conceptions of physical law are addressed head-on in Physics 1250 (e.g. your textbook includes a large number of very helpful ``Pitfall Preventions'' throughout). This is a necessary component of any contemporary introduction to physics, and is addressed in all components of the courses.
\item Students learn the scientific theories that have developed from the 1600s to the present day. They learn different modes of approaching the same phenomena, such as force and energy methods in mechanics.
\item Students will come to understand that Physics 1250 introduces the basic physical laws that underlie all engineering applications (this will include some biomedical engineering applications). Examples of applications are provided in the textbook and in demonstrations in lectures.
\item While it is not the focus of the course, students will come to appreciate some of the social implications of the scientific discoveries discussed in class. The remarkable fact that humankind can understand the motion of objects (even entire planets!) is itself one of the most philosophically important developments in the history of science.
\end{enumerate}

\section*{Official Course Policies for Physics 1250 \& 1251 }

\noindent Available here: \url{http://www.physics.ohio-state.edu/phys1250/general.pdf} \\

\section*{Texts and Required Material}
\begin{itemize}

\item Serway and Jewett, Physics for Scientists and Engineers, 10th Edition eBook.

\item No lab workbook is needed. We will provide you with lab worksheet materials.

%\item Worksheets for Physics 1250 Laboratory; Available at Marion bookstore. Required.

\end{itemize}

\noindent You will have automatic access to an electronic version of the textbook (eBook) through the CarmenBooks program. So instead of going to a bookstore to buy the textbook, the university automatically detects that you have enrolled in the course and a \$40 textbook fee for this course is automatically added to your tuition bill. {\color{BrickRed} YOU DO NOT NEED TO USE A CREDIT CARD TO BUY ANYTHING FROM CENGAGE OR WEBASSIGN}. You do NOT need to purchase a ``webassign access code'' or ``Cengage Unlimited''. Instead, the \$40 fee that is automatically added to your tuition will allow you to access the textbook through webassign as discussed in the next section. \\

\noindent This \$40 fee will give you online (eBook) access to Serway \& Jewett 10th edition. Because of the videos and interactive figures, having a paper textbook is not necessary for this class. \\

\noindent We anticipate that this system will be better than the previous one where you had to purchase a webassign access code. Also worth noting is that the \$40 fee added to your tuition is the lowest price that students have had to pay for textbook access in this course in recent memory. Having this fee wrapped into your tuition also makes it easier to get money back on your taxes.  \\

\noindent More information about the CarmenBooks program is available at \url{affordablelearning.osu.edu/carmenbooks/students} \\

\newpage

\section*{Grading}

\begin{multicols}{2}


\tagpdfsetup{table/header-rows={1,8}}
\begin{tabular}{lcc}
        & Percent & Points \\
\hline
STEM Fluency & 5\% & 40 \\ 
Class Participation & 12.5\% & 100 \\
Homework & 15\% & 120 \\
Exams   & 30\% & 240 \\
Lab work      & 12.5\% & 100 \\
Programming Labs & 10\% & 80 \\
Final   & 15\%  & 120 \\
\hline
Total  & 100\%  & 800 \\
\end{tabular} \\
\\
\\
\begin{center}
    OSU Standard Grading Scale \\
    \label{tab:scale}
    \tagpdfsetup{table/header-rows={0,11}}
    \begin{tabular}{|c|l|}
\hline
93-100: & A  \\
90-92.9: & A- \\
87-89.9: & B+ \\
83-86.9: & B \\
80-82.9: & B- \\
77-79.9: & C+ \\
73-76.9: & C \\
70-72.9: & C- \\
67-69.9: & D+ \\
60-66.9: & D \\
Below 60: & E \\
\hline
    \end{tabular}
\end{center}
\end{multicols}


%\begin{enumerate}

%\item {\bf If you are  not continuing on to Physics 1251} you can purchase ``single-term access'' from webassign. Log in to \url{https://www.webassign.net/osu/login.html} and select ``purchase access online'' then purchase ``Single-term access to homework and eBook''.  

%\item {\bf If you do plan to continue on to Physics 1251} (whether at Marion or Columbus campus) and want to save a little money you can purchase ``Lifetime of edition'' access. Log in to \url{https://www.webassign.net/osu/login.html} and select ``purchase access online'' then purchase ``Lifetime of Edition access to homework and eBook''.

%\end{enumerate}

\section*{Homework / Webassign}

\noindent Homeworks will be completed electronically through webassign which you will be able to access through CarmenCanvas \url{http://osu.instructure.com}  \\

\noindent For a gentle introduction to webassign see  HW\#0. This assignment is totally optional but you may find it very helpful. \\

\noindent  {\color{blue} Homeworks will typically be due at 10pm}. This timing is intended to maximize the availability of the instructors to the students as they complete the homework. \\

\section*{Math Prerequisites}

\noindent Math 1151 (Calculus I) or equivalent (such as AP Calculus) is recommended for the course, but concurrent enrollment in Math 1151 is sufficient for our purposes. There will be some calculus in the course, but the problems will mostly involve trigonometry and algebra. \\

\section*{Programming Labs \& Conventional Labs}

\noindent Physics 1250 at OSU Marion involves both conventional labs as well as programming labs developed by Dr. Chris Orban. %The programming labs can be worked on outside the classroom, but students are encouraged to use time in the lecture section to complete these exercises. 
Note that these programming labs are designed for students with no prior programming experience, and most students find that it is easier to complete than the conventional labs. To complete the programming labs go to \href{http://stemcoding.osu.edu}{stemcoding.osu.edu} and register your e-mail. Use the course join key \texttt{marion1250spring2026}  The next page lists the due dates for these activities as ``Prog. lab. \#1, \#2 ...'' These activities are 1. Planetoids, 2. Lunar Descent, 3. Bellicose birds, 4. Planetoids with momentum and Pong, 5. Planetoids with torque, and 6. Planetoids with a spring.  \\

\section*{Drop \& Attendance policy}

\noindent The lowest homework grade and the lowest Stem Fluency score will be dropped. There are no ``drops'' for the lab grade. Unless you have a legitimate excuse (e.g. sickness, family emergency) you must complete all laboratory activities in order to receive full credit for the labs. Please let an instructor know well in advance if you foresee a time conflict with completing one of the lab sessions. \\

\section*{General Concerns and Resources for Students at OSU Marion}

\noindent The administration of OSU Marion have compiled a list of important information that students should view this list is available here

\noindent \url{https://osumarion.osu.edu/faculty-and-staff/marion-campus-syllabus-statements.html}

\noindent The following items in this syllabus are topics of general information that are not covered in the above link, and the information provided here may assume that you have read through the information at the link above.  \\

\section*{Sexual harassment and Sexual violence}

\noindent The link provided by the administration includes a statement about ``Title IX" which is a set of laws that make it clear that violence and harassment based on sex and gender are Civil Rights offenses subject to the same kinds of accountability and the same kinds of support applied to offenses against other protected categories (e.g., race). \\

\noindent Something that is important to understand but is not entirely clear from the information at that link, is that victims of sexual violence have a right to reach out to and work with local police in an effort to achieve justice even if all parties involved are OSU students or if the incident took place on campus. The University's system to respond to incidents of sexual violence does not negate your rights within the wider US justice system. \\

%\noindent Violence and harassment based on sex and gender are Civil Rights offenses subject to the same kinds of accountability and the same kinds of support applied to offenses against other protected categories (e.g., race).  If you or someone you know has been sexually harassed or assaulted, you can contact OSU Marion's Title IX Coordinator, Shawn Jackson, at \href{mailto:jackson.368@osu.edu}{jackson.368@osu.edu} or you can contact OSU's Title IX office at \href{mailto:titleix@osu.edu}{titleix@osu.edu}. You can also reach out to the instructors of this course, however please be aware that faculty and other personnel are required to report to the University's Title IX Office any instances of sexual violence or harassment that students disclose. If requested, your instructors can put you in touch with counselors who are not mandated reporters who can discuss your case and help you decide whether to contact the Title IX office. See \href{http://titleix.osu.edu}{titleix.osu.edu} for more information. \\

\section*{Credit Hour and Work Expectation}

\noindent This is a 5-credit-hour course. According to Ohio State policy, students should expect around 5 hours per week of time spent on direct instruction (instructor content and Carmen activities, for example) in addition to 10 hours of out of class-time including homework, reading assignments and other activities to receive a grade of (C) average. An excellent guide to scheduling and study expectations is available at this link \url{https://aschonors.osu.edu/preorient/scheduling} \\

%\noindent {\bf Student participation requirements} \\

%\noindent Because this course involves distance-education, your attendance is partly based on your online activity and participation. The following is a summary of everyone's expected participation:
%\begin{enumerate}
%    \item Viewing lectures -- each week there will be two approximately 45 minute lectures for you to watch whenever is convenient. You may watch these lectures at 1.25$\times$ or 1.5$\times$ speed if needed. Some studies show that watching lectures at 2$\times$ speed or more may be too fast for you to adequately follow the topic.
%    \item Worksheets -- Each lecture typically has one worksheet associated with it that you will work with your group (you and typically three other students) to complete. There will often be other activities that are related to the lecture that your group will be asked to complete such as writing your own practice questions for exams and the final
%    \item Other assignments -- you can also work with your group members to complete the homeworks and programming labs in this course although each person will need to turn in separate work. We discourage you from relying on group members to complete Reading Quizzes and Essential Skills questions because these assignments are not intended to be particularly time intensive and we typically provide multiple chances to get the correct answer.
%\end{enumerate}

%\noindent If you have a question that you believe may be of interest to others in the class, please post to the “Ask the instructor” discussion board on Carmen. Office hours are digital via Carmen Zoom.  \\

\section*{Faculty feedback and response time}

\noindent I am providing the following list to give you an idea of my intended availability throughout the course. (Remember that you can call 614-688-HELP at any time if you have a technical problem.) \\

\noindent Grading and feedback -- For large weekly assignments, you can generally expect feedback within 7-10 days. 
E-mail -- I will reply to e-mails within24 hours on school days. \\


\section*{Course technology}

\noindent For help with your password, university e-mail, Carmen, or any other technology issues, questions, or requests, contact the OSU IT Service Desk. Standard support hours are available at \url{https://ocio.osu.edu/help/hours}, and support for urgent issues is available 24x7. \\

\noindent Carmen, Ohio State’s Learning Management System, will be used to host materials and activities throughout this course. To access Carmen, visit \url{http://carmen.osu.edu} or \url{http://osu.instructure.com}.  Log in to Carmen using your name.\# and password. If you have not setup a name.\# and password, visit \url{my.osu.edu} \\

\noindent Help guides on the use of Carmen can be found at \url{https://resourcecenter.odee.osu.edu/carmen}
If you need additional services to use Carmen, please request accommodations with your instructor. Information on the accessibility features of Carmen are described at this link \url{https://community.canvaslms.com/t5/Accessibility/Accessibility-within-Canvas/ba-p/261501} \\

\noindent Carmen Zoom -- virtual course meetings and office hours will be held through Ohio State’s conferencing platform, Carmen Zoom. More information about using Carmen Zoom is available at this link \url{https://resourcecenter.odee.osu.edu/carmenzoom} \\

\section*{Student Services}

\noindent The Student Service Center assists with financial aid matters, tuition and fee payments. Please see their site at: \url{http://ssc.osu.edu} \\

\section*{Tutoring Services}

\noindent Students are strongly encouraged to use the free tutor services available in the Math and Engineering Learning Center which is a part of the Marion Academic Success Center, located on the library's second floor. For more information \url{https://u.osu.edu/mathengineercenter/hours-2/} or email \href{mailto:garapati.8@osu.edu}{garapati.8@osu.edu} \\

\section*{Statement on Copyrighted Material}

\noindent The materials in this course are, unless otherwise noted, copyrighted and owned by Prof. Chris Orban. Uploading any course materials to websites such as chegg.com or coursehero.com for wider distribution is a violation of both the student code of conduct, and Prof. Chris Orban's rights as an author. \\

\section*{Statement on Intellectual Diversity}

\noindent Ohio State is committed to fostering a culture of open inquiry and intellectual diversity within the classroom. This course will cover a range of information and may include discussions or debates about controversial issues, beliefs, or policies. Any such discussions and debates are intended to support understanding of the approved curriculum and relevant course objectives rather than promote any specific point of view. Students will be assessed on principles applicable to the field of study and the content covered in the course. Preparing students for citizenship includes helping them develop critical thinking skills that will allow them to reach their own conclusions regarding complex or controversial matters. \\

\section*{Statement on Artificial Intelligence Tools}

\noindent To maintain a culture of integrity and respect, generative artificial intelligence (AI) tools such as ChatGPT should not be used in the completion of course assignments unless an instructor for a given course specifically authorizes their use. Using generative AI tools outside of an instructor's authorization can be a violation of the student code of conduct, which could result in a referral to the Committee on Academic Misconduct. \textbf{The following is the only acceptable use of generative AI tools in this course: You may use large language models (LLMs) as a part of studying for exams.} \\

\section*{Tentative Course Schedule}

\noindent The course schedule is shown on the next page. The schedule is subject to change. Check the course page on \url{http://osu.instructure.com} for the latest version of the course calendar and deadlines

\newpage 
\begin{center}
\noindent \textbf{\large Tentative Course Schedule} 
\end{center}

%\vspace{0.5cm}

\tagpdfsetup{table/header-rows={1,51}}
\begin{tabular}{|l|c|c|c|c|}
\hline
   & Dates  &  Topic / Item & STEM fluency & Assignments Due\\
\hline
 & Jan. 12 & Introduction/Lecture1,2  & & \\ %lecture1 and part of lecture2
Week 1 & Jan. 14 & Lecture3 &  & \\ %lect3a is vectors, lect3b is gravity
& Jan. 16 & No Lab & SF \#1 &  HW \#0 \\ 
\hline
 & Jan. 19 & {\color{DarkOrchid} No class! MLK Jr. day}   &  &  \\ 
Week 2 & Jan. 21 & Lecture4 &  & HW \#1 \\  %lect4 is Newton's laws
   & Jan. 23 & Lab 1. 1-D Kinematics   &  &   \\
\hline
   & Jan. 25 &  & SF \#2  &   \\
\hline
 & Jan. 26 & Lecture5  &  &  \\ %lect5 is friction and normal force 
Week 3 & Jan. 28 & {\color{DarkOrchid} No class}  &  & \\ 
   & Jan. 30 & Lab 2: 2-D Kinematics  &  & \\
\hline 
   & Feb. 1 &   & SF \#3 & \\
\hline
   & Feb. 2 & Lecture6  &   &  HW \#2 \\   %Lect6 is inclined planes (friction part2)
Week 4 & Feb. 4  & Lecture7 &  &  Prog. Labs \#1 \& \#2 \\ %Lect7 (strings, tension, circular motion)
   & Feb. 6 & Lab 3: Dynamic Forces &  &  \\
\hline
   & Feb. 8 & & SF \#4 &  \\
\hline
 & Feb. 9 & Lecture8  &   &  HW \#3 \\ %Lecture 8 (circular motion part 2) 
Week 5 & Feb. 11 & Lecture9  &  & Prog. Lab \#3  \\   %Lecture 9 (circular motion in the vertical direction)
   & Feb. 13 & Lab 4: Static Friction   & &  \\ 
\hline
   & Feb. 15 & & SF \#5 &  \\
\hline
 & Feb. 16 & Lecture10 & &  HW \#4a  \\   %Lecture 10 (circular motion at an angle and putting it all together)
Week 6   & Feb. 18 & Exam Review & & HW \#4b \\ %lecture 11 (midterm review)
& Feb. 20 &  {\color{BrickRed} Exam \#1}   &  &  \\   
\hline
& Feb. 23 & Lecture12   &   &   \\ % Lecture 12  work and energy
Week 7 & Feb. 25 & Lecture13 &  &   \\ % lecture 13 Drag Racing and Power 
& Feb. 27 &  Lab 5: Conservation of Energy &   & HW \#5 \\ 
\hline
   & Mar. 1 & & SF \#6  &  \\
\hline
 & Mar. 2 & Lecture14   &  &  \\   %Lecture14 Momentum
Week 8 & Mar. 4 & Lecture15  &  & HW \#6  \\  %Lecture 15 momentum part 2
       & Mar. 6 & Lab 6: Energy and Momentum &  &  Prog. Lab \#4a (Pong)  \\
\hline
       & Mar. 8 & & SF \#7  &  \\
\hline
    & Mar. 9  & Lecture16  &  &  Prof. Lab \#4b (momentum)   \\  %Lecture 16 momentum 3 and rotation
Week 9 & Mar. 11 & Lecture17 &  & HW \#7 \& \#8 \\ % Lecture 17 torque part 2
     & Mar. 13 & {\color{DarkOrchid} No class!} &  & \\
\hline
Week 10 & Mar. 16-20 & {\color{DarkOrchid} Spring Break!} &  & \\
\hline
   & Mar. 23 & Lecture18 & & Prog. Lab \#5 \\ %Lecture 18 (Angular momentum)
Week 11 & Mar. 25 & Midterm review &   & HW \#9  \\  % Lecture 19 (Midterm 2 review)
   & Mar. 27 & Lab 7: Rotational Dynamics &  & \\
 \hline
   & Mar. 30 & {\color{BrickRed} Exam \#2}    &  &  \\  %Midterm 2
Week 12 & Apr. 1 & Lecture20 &   &    \\ %Lecture 20 (oscillations)
   & Apr. 3 & Lab 8. Vibrations Lab & &  \\ 
\hline
   & Apr. 6 &  Lecture21  &  & HW \#10 \\ %Lecture 21 (damped oscillations)   
Week 13 & Apr. 8 &  Lecture22  &  & Prog. Lab. 6 \\ % Lecture 22 (Fluid Mechanics)
   & Apr. 10 &  {\color{DarkOrchid} No class!} &    &  \\ 
   \hline
   & Apr. 13 &   Lecture23  &    & HW \#11  \\ %Lecture 23 (Fluid Mech part 2) 
Week 14 &  Apr. 15 &   Lecture24  &    & \\  %Lecture 24 (Thermodynamics part 1)
   &  Apr. 17 & Lab 9. Barometer & &  \\ 
   \hline
   & Apr. 20 & Lecture25 &  & HW \#12 \\  %Lecture 25 (Thermodynamics part 2, potato cannon)
Week 15 & Apr. 22 & HW help / Final Review &  & HW \#13  \\  
   & Apr. 24 & Lab 10: Fluid Mechanics  &    &  \\
\hline
  & Apr. 27 & Final Exam Review & SF review &   \\ 
\hline 
 & May 4 & {\color{BrickRed} Final Exam: 8:00am}  &  & \\
\hline
\end{tabular} 
%\noindent {\color{blue}*All chapters are from the 10th edition of ``Physics for Scientists and Engineers'' by Serway \& Jewett unless otherwise noted.} \\
\end{document}


